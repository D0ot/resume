% !TEX TS-program = xelatex
% !TEX encoding = UTF-8 Unicode
% !Mode:: "TeX:UTF-8"

\documentclass{resume}
\usepackage{zh_CN-Adobefonts_external} % Simplified Chinese Support using external fonts (./fonts/zh_CN-Adobe/)
% \usepackage{NotoSansSC_external}
% \usepackage{NotoSerifCJKsc_external}
% \usepackage{zh_CN-Adobefonts_internal} % Simplified Chinese Support using system fonts
\usepackage{linespacing_fix} % disable extra space before next section
\usepackage{cite}

\begin{document}
\pagenumbering{gobble} % suppress displaying page number

\name{耿宝源}

\basicInfo{
  \email{gugegby@gmail.com} \textperiodcentered\ 
  \phone{(+86) 185-397-99675} \textperiodcentered\ 
  \github[d0ot]{https://github/D0ot}}
 
\section{\faGraduationCap\  教育背景}
\datedsubsection{\textbf{合肥工业大学}, 宣城, 安徽}{2009 -- 2013}
\textit{学士}\ 计算机科学与技术专业

\section{\faUsers\ 实习/项目经历}
\datedsubsection{\textbf{合肥工业大学RoboCup实验室}}{2019年4月 -- 至今}
\role{3D仿真开发小组成员}{}
\begin{onehalfspacing}
合肥工业大学RoboCup实验室主要使用C++/Python进行机器人仿真模拟的相关开发。
\begin{itemize}
  \item 维护RoboCup 3D仿真开发环境的一键安装脚本。
  \item 在C++环境使用CMA-ES算法对智能体行为进行优化。
\end{itemize}
\end{onehalfspacing}

\datedsubsection{\textbf{GLEC2}}{2021年6月 -- 2021年7月}
\role{个人项目,计算机体系结构的课程设计}{}
\begin{onehalfspacing}
一个五级流水线的RV32I的CPU核心设计,支持通过Bypass和Stall解决Data Hazard。实现了非特权的几乎全部指令,GCC编译出的机器码可以直接在CPU上运行。
\begin{itemize}
  \item 经典五级流水线设计
  \item 完整的RV32I非特权级指令集
  \item 可配置,通过改变参数可以配置CPU核心的参数。
  \item 使用SpinalHDL(一个基于Scala的DSL)写成,敏捷开发实践。
\end{itemize}
\end{onehalfspacing}

\datedsubsection{\textbf{ZGMSOS}}{2021年6月 -- 至今}
\role{竞赛项目}{}
\begin{onehalfspacing}
一个XV6风格的玩具内核实现,是系统能力培养大赛操作系统内核赛的初赛参赛作品。项目仍在开发中。
\begin{itemize}
  \item 支持多进程 / 上下文切换
  \item 可以加载ELF文件,运行用户程序
  \item 相比于XV6,有着更小细粒度的内存管理
  \item 一个只读的FAT32文件系统
  \item 一个简单的虚拟文件系统
  \item 支持一些系统调用
\end{itemize}
\end{onehalfspacing}

\datedsubsection{\textbf{其他项目}}{}
\role{个人项目,个人实用工具}{}
\begin{onehalfspacing}
下面是我的一些个人小项目。
\begin{itemize}
  \item GLEC: 一个单周期的RV32I指令集CPU核心实现
  \item Robocup3dInstaller: RoboCup3D仿真开发环境安装脚本
  \item gb\_onenotehelper: 一个增强触控笔上按键功能的AHK脚本,适用于单按键笔Windows设备
  \item subw2r: 一个命令行V2ray订阅小工具,输入定订阅链接,生成config.json配置文件。
\end{itemize}
\end{onehalfspacing}

\begin{onehalfspacing}
\textbf{上面提到的主要项目都已经Pin到我的Github个人页面了}
\end{onehalfspacing}

% Reference Test
%\datedsubsection{\textbf{Paper Title\cite{zaharia2012resilient}}}{May. 2015}
%An xxx optimized for xxx\cite{verma2015large}
%\begin{itemize}
%  \item main contribution
%\end{itemize}

\section{\faCogs\ 技能}
% increase linespacing [parsep=0.5ex]
\begin{itemize}[parsep=0.5ex]
  \item 编程语言: C/C++, Python > Bash, Java, Rust, Scala
  \item 平台: Linux
  \item 开发: Linux环境编程,裸机环境编程
  \item 其他: 对Linux命令行有一定掌握,因为日用系统是Archlinux。熟练使用Vim。

\end{itemize}


\section{\faHeartO\ 获奖情况}
\datedline{\textit{二等奖}, 2020 中国大学生数学建模大赛本科组}{2020 年11 月}
\datedline{\textit{一等奖}, 2020 安徽省大数据与人工智能应用竞赛本科组}{2020 年 10月}
\datedline{\textit{一等奖}, 2020 安徽省机器人大赛,RoboCup仿真3D比赛}{2020 年 10月}
\datedline{\textit{H奖}, 2020美国大学生数学建模竞赛}{2020年4月}
\datedline{\textit{一等奖}, 合肥工业大学2018-2019学年奖学金}{2019年9月}


\section{\faInfo\ 其他}
% increase linespacing [parsep=0.5ex]
\begin{itemize}[parsep=0.5ex]
  \item GitHub: https://github.com/d0ot
  \item 语言: 英语 - 中等(CET6 - 528, CET4 - 561)
\end{itemize}

%% Reference
%\newpage
%\bibliographystyle{IEEETran}
%\bibliography{mycite}
\end{document}
