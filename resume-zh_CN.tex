% !TEX TS-program = xelatex
% !TEX encoding = UTF-8 Unicode
% !Mode:: "TeX:UTF-8"

\documentclass{resume}
\usepackage{zh_CN-Adobefonts_external} % Simplified Chinese Support using external fonts (./fonts/zh_CN-Adobe/)
% \usepackage{NotoSansSC_external}
% \usepackage{NotoSerifCJKsc_external}
% \usepackage{zh_CN-Adobefonts_internal} % Simplified Chinese Support using system fonts
\usepackage{linespacing_fix} % disable extra space before next section
\usepackage{cite}

\begin{document}
\pagenumbering{gobble} % suppress displaying page number

\name{耿宝源}

\basicInfo{
  \email{gengbaoyuan@gmail.com} \textperiodcentered\ 
  \phone{(+86) 185-397-99675} \textperiodcentered\ 
  \github[d0ot]{https://github/D0ot}}
 
\section{\faGraduationCap\  教育背景}
\datedsubsection{\textbf{合肥工业大学}, 宣城, 安徽}{2018年9月 -- 2022年6月}
计算机科学与技术专业,本科

\section{\faUsers\ 工作经历}
\datedsubsection{\textbf{深信服科技股份有限公司}, 深圳}{2022年7月 -- 2024年9月}
\role{后台软件开发工程师}{云技术中台部}
\begin{onehalfspacing}
  HCI 超融合产品线,围绕Linux内核、QEMU等虚拟化组件进行相关功能开发、问题定位、性能优化
\begin{itemize}
  \item NVMe-oF 与 SangforFS 文件系统的适配预研,在NVMe驱动中新增 NVMe Fused Command 支持
  \item 为 SPDK bdev 增加 I/O 重试功能,满足业务设计需求
  \item QEMU、Linux内核、用户态程序的问题定位与一般性能优化调优
  \item CI/CD 维护,共享的Neovim配置维护,开发环境部署脚本维护等。其他的一些Glue Work
\end{itemize}
\end{onehalfspacing}

% Reference Test
%\datedsubsection{\textbf{Paper Title\cite{zaharia2012resilient}}}{May. 2015}
%An xxx optimized for xxx\cite{verma2015large}
%\begin{itemize}
%  \item main contribution
%\end{itemize}

\section{\faCogs\ 技能}
% increase linespacing [parsep=0.5ex]
\begin{itemize}[parsep=0.5ex]
  \item 编程语言:C/C++,Bash,Python
  \item 平台相关:熟悉 Linux 用户态相关内容,了解发行版构建。熟悉systemd。非常熟悉Arch Linux发行版,了解Red Hat系发行版
  \item 硬件相关:熟悉 X86\_64体系结构; 了解PCIe体系结构,部分了解NVMe协议
  \item 网络相关:掌握一般的TCP/IP网络知识; 熟悉 RoCEv2 的网络配置与交换机配置,熟读相关经典论文
  \item 虚拟化:了解内存虚拟化实现,熟悉Virtio与VFIO;了解不同平台上的PV特性。例如KVM的PV特性与Hyper-V Enlightenment
  \item 性能优化:了解Linux下相关性能分析和优化的方法,熟悉x86\_64下在微架构层面的性能分析优化,TMA性能分析方法
  \item 问题定位: 能够在Linux平台下使用 bpftrace gdb crash drgn 等工具完成从用户态程序到内核的问题定位、了解KASAN,Kfence,pstore等调试手段
  \item Linux内核相关:了解伙伴系统的内存分配释放流程与内存回收
\end{itemize}


\section{\faHeartO\ 获奖情况}
\datedline{深信服科技股份有限公司,2022年度优秀新人}{2023 年2 月}
\datedline{深信服科技股份有限攻宠,2023年上半年优秀员工}{2023 年9 月}


\section{\faInfo\ 其他}
% increase linespacing [parsep=0.5ex]
\begin{itemize}[parsep=0.5ex]
  \item 英语水平 CET6 528 - 能够熟练阅读、撰写英文文档。擅长阅读各类Spec文档
  \item Linux桌面环境用户,Arch Linux用户。Neovim用户
  \item 会定期关注LWN.net和每年的LPC与CLK,也会关注计算机系统会议(例如OSDI ATC FAST)最新的论文, 周末时间会阅读一些论文
  \item 爱好:PC游戏、嵌入式与FPGA Zynq、偶尔会调试开发板硬件驱动和修改DeviceTree
  \item GitHub: https://github.com/d0ot

\end{itemize}

%% Reference
%\newpage
%\bibliographystyle{IEEETran}
%\bibliography{mycite}
\end{document}
